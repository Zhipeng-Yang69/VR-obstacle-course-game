\documentclass{vgtc}                          % final (conference style)
%\documentclass[review]{vgtc}                 % review
%\documentclass[widereview]{vgtc}             % wide-spaced review
%\documentclass[preprint]{vgtc}               % preprint
%\documentclass[electronic]{vgtc}             % electronic version

%% Uncomment one of the lines above depending on where your paper is
%% in the conference process. ``review'' and ``widereview'' are for review
%% submission, ``preprint'' is for pre-publication, and the final version
%% doesn't use a specific qualifier. Further, ``electronic'' includes
%% hyperreferences for more convenient online viewing.

%% Please use one of the ``review'' options in combination with the
%% assigned online id (see below) ONLY if your paper uses a double blind
%% review process. Some conferences, like IEEE Vis and InfoVis, have NOT
%% in the past.

%% Figures should be in CMYK or Grey scale format, otherwise, colour 
%% shifting may occur during the printing process.

%% These few lines make a distinction between latex and pdflatex calls and they
%% bring in essential packages for graphics and font handling.
%% Note that due to the \DeclareGraphicsExtensions{} call it is no longer necessary
%% to provide the the path and extension of a graphics file:
%% \includegraphics{diamondrule} is completely sufficient.
%%
\ifpdf%                                % if we use pdflatex
  \pdfoutput=1\relax                   % create PDFs from pdfLaTeX
  \pdfcompresslevel=9                  % PDF Compression
  \pdfoptionpdfminorversion=7          % create PDF 1.7
  \ExecuteOptions{pdftex}
  \usepackage{graphicx}                % allow us to embed graphics files
  \DeclareGraphicsExtensions{.pdf,.png,.jpg,.jpeg} % for pdflatex we expect .pdf, .png, or .jpg files
\else%                                 % else we use pure latex
  \ExecuteOptions{dvips}
  \usepackage{graphicx}                % allow us to embed graphics files
  \DeclareGraphicsExtensions{.eps}     % for pure latex we expect eps files
\fi%

%% it is recomended to use ``\autoref{sec:bla}'' instead of ``Fig.~\ref{sec:bla}''
\graphicspath{{figures/}{pictures/}{images/}{./}} % where to search for the images

\usepackage{microtype}                 % use micro-typography (slightly more compact, better to read)
\PassOptionsToPackage{warn}{textcomp}  % to address font issues with \textrightarrow
\usepackage{textcomp}                  % use better special symbols
\usepackage{mathptmx}                  % use matching math font
\usepackage{times}                     % we use Times as the main font
\renewcommand*\ttdefault{txtt}         % a nicer typewriter font
\usepackage{cite}                      % needed to automatically sort the references
\usepackage{tabu}                      % only used for the table example
\usepackage{booktabs}                  % only used for the table example
%% We encourage the use of mathptmx for consistent usage of times font
%% throughout the proceedings. However, if you encounter conflicts
%% with other math-related packages, you may want to disable it.


%% If you are submitting a paper to a conference for review with a double
%% blind reviewing process, please replace the value ``0'' below with your
%% OnlineID. Otherwise, you may safely leave it at ``0''.
\onlineid{0}

%% declare the category of your paper, only shown in review mode
\vgtccategory{Research}

%% allow for this line if you want the electronic option to work properly
\vgtcinsertpkg

%% In preprint mode you may define your own headline.
%\preprinttext{To appear in an IEEE VGTC sponsored conference.}

%% Paper title.

\title{Virtual Reality Obstacle Course}

%% This is how authors are specified in the conference style

%% Author and Affiliation (single author).
%%\author{Roy G. Biv\thanks{e-mail: roy.g.biv@aol.com}}
%%\affiliation{\scriptsize Allied Widgets Research}

%% Author and Affiliation (multiple authors with single affiliations).
%%\author{Roy G. Biv\thanks{e-mail: roy.g.biv@aol.com} %
%%\and Ed Grimley\thanks{e-mail:ed.grimley@aol.com} %
%%\and Martha Stewart\thanks{e-mail:martha.stewart@marthastewart.com}}
%%\affiliation{\scriptsize Martha Stewart Enterprises \\ Microsoft Research}

%% Author and Affiliation (multiple authors with multiple affiliations)
\author{Alberto Chan Liu\thanks{e-mail: albertochanliu2018@gmail.com}\\ %
         %
\and Irene Zhang\thanks{e-mail: irenezhang342@gmail.com}\\ %
      %
\and Zhipeng Yang\thanks{e-mail: yangzhipeng666@gmail.com}\\ %
     }

%% A teaser figure can be included as follows, but is not recommended since
%% the space is now taken up by a full width abstract.
%\teaser{
%  \includegraphics[width=1.5in]{sample.eps}
%  \caption{Lookit! Lookit!}
%}

%% Abstract section.
\abstract{This project will provide a playable environment using a VR set, specifically the Quest 2, in which the user will be able to interact with a virtual world as an animated character and enjoy the different challenges the environment will propose to them. The challenges will consist of a set of obstacles different to each other and specific to the different environments. The user will have to pass through all obstacles by jumping, running or moving in order to get to the end of each of the virtual worlds and win each level environment.%
} % end of abstract

%% ACM Computing Classification System (CCS). 
%% See <http://www.acm.org/about/class> for details.
%% We recommend the 2012 system <http://www.acm.org/about/class/class/2012>
%% For the 2012 system use the ``\CCScatTwelve'' which command takes four arguments.
%% The 1998 system <http://www.acm.org/about/class/class/2012> is still possible
%% For the 1998 system use the ``\CCScat'' which command takes four arguments.
%% In both cases the last two arguments (1998) or last three (2012) can be empty.

\CCScatlist{
  \CCScatTwelve{Virtual reality}{Game design}{Obstacle course}{Interaction};
}

%\CCScatlist{
  %\CCScat{H.5.2}{User Interfaces}{User Interfaces}{Graphical user interfaces (GUI)}{};
  %\CCScat{H.5.m}{Information Interfaces and Presentation}{Miscellaneous}{}{}
%}

%% Copyright space is enabled by default as required by guidelines.
%% It is disabled by the 'review' option or via the following command:
% \nocopyrightspace

%%%%%%%%%%%%%%%%%%%%%%%%%%%%%%%%%%%%%%%%%%%%%%%%%%%%%%%%%%%%%%%%
%%%%%%%%%%%%%%%%%%%%%% START OF THE PAPER %%%%%%%%%%%%%%%%%%%%%%
%%%%%%%%%%%%%%%%%%%%%%%%%%%%%%%%%%%%%%%%%%%%%%%%%%%%%%%%%%%%%%%%%

\begin{document}

%% The ``\maketitle'' command must be the first command after the
%% ``\begin{document}'' command. It prepares and prints the title block.

%% the only exception to this rule is the \firstsection command
\firstsection{Introduction}

\maketitle

%% \section{Introduction} %for journal use above \firstsection{..} instead
This game idea is the result of a common interest in our group of understanding the enjoyment from interaction of the user in a game using VR and the different methods used to achieve this goal. All members of the group have tried gaming with technology and found ourselves interested in applying VR in this setting to create our own interactive environment and challenging ourselves to create an enjoyable game for the user. 							
Our objective is to use Unity to create our own different scenes in which a character controlled by the user will have to get to the end of the map avoiding different obstacles along the way by jumping and running away from the objects created to stop the character.


\section{Related Work}

Previous researches like “Evaluating enjoyment, presence, and emulator sickness in VR games based on first- and third- person viewing perspectives”  by Diego Monteiro, Hai-Ning Liang, Wenge Xu, Marvin Brucker, Vijayakumar Nanjappan and Yong Yue
 have shown that when using VR headset in a virtual environment game based on a first-person perspective can lead to simulator sickness and discomfort from the user. Whereas in a third person perspective it is less likely that simulator sickness occurs. However in a third person display, the immersiveness of the game is lost.
Our group was put in the dilemma of doing a third person perspective or a first person perspective. We wanted to make an enjoyable game for all users and not worry about motion sickness, but the virtual reality aspect and immersiveness that we were looking for for the project was not going to be there. Since our project is still in development we are still in discussion on what would be the best solution for this problem.

\subsection{Research Links}
We do research about how to insert a character from Mixamo to Unity 3D. Here is the link for Mixamo: https://www.mixamo.com/#/?page=1&type=Motion%2CMotionPack. 
We also figure out how to build and run our scenes on Android platform with headset: https://youtu.be/JeFHgAblEAk
Obstacles asset we used for our project: https://assetstore.unity.com/packages/templates/packs/obstacle-course-pack-178169
Asset for scenes build: https://assetstore.unity.com/packages/3d/environments/historic/polylised-medieval-desert-city-94557

\section{Methodology}

\begin{itemize}
\item The obstacle game has to be played in a safe environment since the user will be immersed in a virtual world and will not have visual contact with the real world. The game will be playable with a Quest 2 VR headset and the user will feel like he or she is inside an virtual world created by our team members trying to get to the end of the level. With the Quest 2 controller buttons to move the character around and interact with the virtual world objects and obstacles  of the level environment using the different actions the character will provide. [Alberto Chan Liu]
\item There are some interactions used in desert city scene.for current process,users can control character moving,turning and jumping by keyboard. After we set up the headset, users are able to control their moving direction by VR headset.When the character and obstacles collide, the character will fall down and they need to take time to stand up and continue the game.For input devices, we use head tracking to control the moving direction of the character. In addition, the Jump controller is used for our character jumping. Users can adjust the speed and jumping height by physical button controllers.[Zhipeng Yang] 
\item Throughout the design phase of our VR game, many interaction methods were employed to create a fuller player experience. For interactions, the end goal of our project is to have players complete obstacle challenges in virtual reality through some minimal physical movements. For reference frames, the forest level is intended to use the torso reference frame which moves with both physical and virtual rotation. In terms of user locomotion, we plan to have them walk in place while also having control to character movement through the controller. We aim to minimize user moving range but add as many in-place movements as possible for safe considerations. We use controllers over bare hands because of their features of reliability and capability of adding haptics. [Irene Zhang]

\end{itemize}

\section{User Evaluation and Results}

The obstacle game will be created using Unity, it will be implemented using assets from Unity Store and other different web resources so that the game will be more interactive and user friendly. Assets will provide specific shape objects related to the different worlds created, as well as different animations needed and controls over objects and characters.
Each member of the team will be in charge of creating one different level so that the user can choose between them. Each of the virtual worlds will have a theme desert, city, and forest and every theme will have obstacles associated with them. The playable character will be also associated with the theme it is playing in.


\section{Discussions and Conclusions}

\subsection{Future Work}

For next steps, we will set up our VR headset and test our scenes, and figure out how to apply the interactions when the character is hit by obstacles. Furthermore, we have three different difficulty level scenes, and we need to find out how to make the user get into the next scene after they pass the previous scene.
A timer is also on the works to add more interactivity and difficulty to the game, as well as a starting menu in which the user could select the environment or difficulty he or she wants to play in.


%\bibliographystyle{abbrv}
\bibliographystyle{abbrv-doi}
%\bibliographystyle{abbrv-doi-narrow}
%\bibliographystyle{abbrv-doi-hyperref}
%\bibliographystyle{abbrv-doi-hyperref-narrow}

\bibliography{template}
\end{document}
